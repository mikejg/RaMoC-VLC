\documentclass[12pt, a4paper]{scrreprt}
\usepackage[utf8]{inputenc}
\usepackage[T1]{fontenc}
\usepackage{ngerman}
\pagestyle{empty}

\begin{document}
\thispagestyle{empty}

\chapter*{Pizzateig}

Der Pizzateig reicht für \textbf{drei große Bleche}, also entweder was wenn die Familie zu Besuch kommt oder wenn ein paar Nerds mit Brettspielen vor der Türe stehen. Allerdings sollte der Teig schon 2-3 Stunden vorher gemacht werden.

\section*{Zutaten}
Die folgenden Zutaten sollte man immer in der Küche haben. Bis auf die Trockenhefe ist das auch meistens der Fall. Vorsicht: Nichts ist ärgerlicher als eine abgelaufene Packung Trockenhefe die den ganzen Pizzateig ruiniert, also immer schön aufs Haltbarkeitsdatum schauen.

\begin{itemize}
  \item{500 Gram Mehl}
  \item{Eine gute Briese Salz}
  \item{Ein Päckchen Trockenhefe}
  \item{10 Eßlöffel Öl}
  \item{ungefähr 300 ml Wasser}
\end{itemize}

Für den Belag gibt es natürlich keine Vorgaben. Fix sind hier nur die Bestandteile \textbf{Tomatenmark} und \textbf{Käse}. Hier geht eigentlich fast alles, zur Not tut es auch Ketchup und ein Parmesanpäckchen aus der Miracollipackung.


\section*{Zubereitung}


\end{document}

